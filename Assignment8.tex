%\documentclass[11pt,reqno]{amsart}
\documentclass[11pt,reqno]{article}
\usepackage[margin=.8in, paperwidth=8.5in, paperheight=11in]{geometry}
%\usepackage{geometry}                % See geometry.pdf to learn the layout options. There are lots.
%\geometry{letterpaper}                   % ... or a4paper or a5paper or ... 
%\geometry{landscape}                % Activate for for rotated page geometry
%\usepackage[parfill]{parskip}    % Activate to begin paragraphs with an empty line rather than an indent7
\usepackage{graphicx}
\usepackage{pstricks}
\usepackage{amssymb}
\usepackage{epstopdf}
\usepackage{amsmath}
\usepackage{subfigure}
\usepackage{caption}
\pagestyle{plain}
%\renewcommand{\topfraction}{0.3}
%\renewcommand{\bottomfraction}{0.8}
%\renewcommand{\textfraction}{0.07}
\DeclareGraphicsRule{.tif}{png}{.png}{`convert #1 `dirname #1`/`basename #1 .tif`.png}

\title{Real Analysis $\mathbb{I}$: \\ Assignment 8}
\author{Andrew Rickert}
\date{Started: June 20, 2011 \\ \hspace{1pt} Ended: June ??,  2011}                                           % Activate to display a given date or no date

\begin{document}
\maketitle


% Page 1
\begin{flushleft} 
\textbf{Class 18.100B} - Problem 1\\
\rule{500pt}{1pt}\\
\end{flushleft} 

We need to show that if $f$ is continuous on $\mathbb{R}^+$ and $f(x^2) = f(x)$ for all $x \in \mathbb{R}^+$ then $f$ is constant.\\
\indent We break the domain of the problem up into $D_1 = [0,1)$ and $D_2 = [1,\infty)$. We also need two theorems from Rudin which are:


\begin{equation}
\lim_{n \to \infty} p^n = 0 \quad \text{if} \; p < 1 \label{eqn:limtozero}
\end{equation}

\begin{equation}
\lim_{n \to \infty} \sqrt[n]{p} = 1 \quad \text{if} \; p > 0 \label{eqn:limtoone}
\end{equation}


For any $x \in D_1$ we may form a sequence $(x^n)$, and by equation $(\ref{eqn:limtozero})$, we know that $\lim_{n \to \infty} x^n = 0$. Then, by the continuity of $f$ we have $\lim_n f(x^n) = f(0)$.

\vspace{15pt}
\begin{flushleft} 
\textbf{Class 18.100B} - Problem 2\\
\rule{500pt}{1pt}\\
\end{flushleft} 


\vspace{15pt}
\begin{flushleft} 
\textbf{Class 18.100B} - Problem 3\\
\rule{500pt}{1pt}\\
\end{flushleft} 



\vspace{15pt}
\begin{flushleft} 
\textbf{Class 18.100B} - Problem 4\\
\rule{500pt}{1pt}\\
\end{flushleft} 


\vspace{15pt}
\begin{flushleft} 
\textbf{Class 18.100B} - Problem 5\\
\rule{500pt}{1pt}\\
\end{flushleft} 


\vspace{15pt}
\begin{flushleft} 
\textbf{Class 18.100B} - Problem 6\\
\rule{500pt}{1pt}\\
\end{flushleft} 


\vspace{15pt}
\begin{flushleft} 
\textbf{Class 18.100B} - Problem 7\\
\rule{500pt}{1pt}\\
\end{flushleft} 


%\vspace{15pt}
%\begin{flushleft} 
%\textbf{Class 18.100B} - Extra Problem 1\\
%\rule{500pt}{1pt}\\
%\end{flushleft} 


\end{document}  