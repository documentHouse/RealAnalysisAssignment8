%\documentclass[11pt,reqno]{amsart}
\documentclass[11pt,reqno]{article}
\usepackage[margin=.8in, paperwidth=8.5in, paperheight=11in]{geometry}
%\usepackage{geometry}                % See geometry.pdf to learn the layout options. There are lots.
%\geometry{letterpaper}                   % ... or a4paper or a5paper or ... 
%\geometry{landscape}                % Activate for for rotated page geometry
%\usepackage[parfill]{parskip}    % Activate to begin paragraphs with an empty line rather than an indent7
\usepackage{graphicx}
\usepackage{pstricks}
\usepackage{amssymb}
\usepackage{epstopdf}
\usepackage{amsmath}
\usepackage{subfigure}
\usepackage{caption}
\pagestyle{plain}
%\renewcommand{\topfraction}{0.3}
%\renewcommand{\bottomfraction}{0.8}
%\renewcommand{\textfraction}{0.07}
\DeclareGraphicsRule{.tif}{png}{.png}{`convert #1 `dirname #1`/`basename #1 .tif`.png}

\title{Real Analysis $\mathbb{I}$: \\ Assignment 8}
\author{Andrew Rickert}
\date{Started: June 20, 2011 \\ \hspace{1pt} Ended: June ??,  2011}                                           % Activate to display a given date or no date

\begin{document}
\maketitle


% Page 1
\begin{flushleft} 
\textbf{Class 18.100B} - Problem 1\\
\rule{500pt}{1pt}\\
\end{flushleft} 

We need to show that if $f$ is continuous on $\mathbb{R}^+$ and $f(x^2) = f(x)$ for all $x \in \mathbb{R}^+$ then $f$ is constant.\\
\indent We break the domain of the problem up into $D_1 = [0,1)$ and $D_2 = [1,\infty)$. We also need two theorems from Rudin which are:


\begin{equation}
\lim_{n \to \infty} p^n = 0 \quad \text{if} \; p < 1 \label{eqn:limtozero}
\end{equation}

\begin{equation}
\lim_{n \to \infty} \sqrt[n]{p} = 1 \quad \text{if} \; p > 0 \label{eqn:limtoone}
\end{equation}


For any $x \in D_1$ we may form a sequence $(x^n)$, and by equation $(\ref{eqn:limtozero})$, we know that $\lim_{n \to \infty} x^n = 0$. Then, by the continuity of $f$ we have $\lim_{n \to \infty} f(x^n) = f(0)$. Now, by hypothesis we have that \\
$f(x^2) = f(x)$, but $f(x^4) = f(x^2)$ so $f(x^4) = f(x)$. If we continue in this way we get a sequence $(f(x^{2n}))$ such that $f(x^{2n}) = f(x)$ for all $n \in \mathbb{N}$. By the previous comments we get $f(x) = \lim_{n \to \infty} f(x^{2n}) = f(0)$.\\
\indent Now suppose that $x \in D_2$. Since $1 \le x \implies x \le x^2 \implies \sqrt{x} \le x$ and consider again the relation $f(x^2) = f(x)$. Let $y \in D_2$ then let $y = x^2$ then we have $f(y) = f(\sqrt{y})$ for any $y \in D_2$. By reapplying the relation to a $y \in D_2$ we get a sequence $(f(\sqrt[2n]{y}))$ and by $(\ref{eqn:limtoone})$ we can again use continuity to say that $f(y) = \lim_{n \to \infty}f(\sqrt[2n]{y}) = f(1)$. We have shown that $f(x) = f(1)$ for all $x \in \mathbb{R}^+$.

\vspace{15pt}
\begin{flushleft} 
\textbf{Class 18.100B} - Problem 2\\
\rule{500pt}{1pt}\\
\end{flushleft} 

Given that $f$ is defined for $x > 0$ and differentiable and $f'(x) \to 0$ as $x \to \infty$ show that \\
$g(x) = f(x+1) - f(x)$ has the property that $g(x) \to 0$.  \\
\indent The proof is based on the mean value theorem which states that there exists an $x' \in (a,b)$ such that $f(b)-f(a) = f'(x')(b-a)$. If we let $a = x$ and $b = x+1$ then we get 
\[ g(x) = f(x+1) - f(x) = f'(x')(x+1 - x) = f'(x') \]
Since $x' \in (x,x+1)$ for all $x$ we have $x < x'$ so 
\[ \lim_{x \to \infty} g(x) = \lim_{x \to \infty} f'(x')  =   \lim_{x' \to \infty} f'(x') = 0\]

\newpage
\vspace{15pt}
\begin{flushleft} 
\textbf{Class 18.100B} - Problem 3\\
\rule{500pt}{1pt}\\
\end{flushleft} 

We are being asked to show that the following equation has at least one root between 0 and 1:
\[ C_0 + C_1 x + \cdots + C_{n-1} x^{n-1} + C_n x^n = 0 \] 
 given that 
 \begin{equation} 
 C_0 + \frac{C_1}{2} + \cdots +\frac{C_{n-1}}{n} + \frac{C_n}{n+1}  = 0 \label{eqn:dercond}
 \end{equation}
 
 \indent Given the form of $(\ref{eqn:dercond})$ it is clear we should look at an auxiliary function to solve the problem. Consider the function:
 \[ g(x) = C_0 x + C_1 \frac{x^2}{2} + \cdots + C_{n-1} \frac{x^n}{n} + C_n \frac{x^{n+1}}{n+1} \]
 
\vspace{15pt}
\begin{flushleft} 
\textbf{Class 18.100B} - Problem 4\\
\rule{500pt}{1pt}\\
\end{flushleft} 


\vspace{15pt}
\begin{flushleft} 
\textbf{Class 18.100B} - Problem 5\\
\rule{500pt}{1pt}\\
\end{flushleft} 


\vspace{15pt}
\begin{flushleft} 
\textbf{Class 18.100B} - Problem 6\\
\rule{500pt}{1pt}\\
\end{flushleft} 


\vspace{15pt}
\begin{flushleft} 
\textbf{Class 18.100B} - Problem 7\\
\rule{500pt}{1pt}\\
\end{flushleft} 


%\vspace{15pt}
%\begin{flushleft} 
%\textbf{Class 18.100B} - Extra Problem 1\\
%\rule{500pt}{1pt}\\
%\end{flushleft} 


\end{document}  